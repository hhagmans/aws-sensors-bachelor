In diesem Kapitel werden die genutzten Amazon Web Services Kinesis und DynamoDB betrachtet und deren Nutzen f�r ein Projekt dieser Art evaluiert. Zudem wird der potentielle Nutzen von AWS IoT bei Nutzung echter Sensoren evaluiert und noch einige andere Implementierungsm�glichkeiten betrachtet.
\section{Evaluation Kinesis und DynamoDB}
Amazon Kinesis war in diesem Projekt eine sehr wichtige Komponente, die es erm�glichte das Projekt schnell durchzuf�hren, aber auch gute Ergebnisse erzielen zu k�nnen. Kinesis ist leicht nutzbar und bietet einen guten Durchsatz der Daten. So war es ohne Probleme m�glich mehrere tausend Temperaturdaten von verschiedenen Sensoren pro Sekunde �ber Amazon Kinesis zu �bertragen und zu verarbeiten. Dabei wurde zur Vereinfachung ein einfacher kommaseparierter String erzeugt, der die Temperaturdaten enthielt, und mittels Amazon Kinesis �bertragen. Dieser String wurde vom Consumer verarbeitet und die Temperaturdaten dann manuell in DynamoDB persistiert. Es ist aber auch m�glich, eigene Modelklassen f�r die Daten zu erstellen und mittels Annotationen direkt anzugeben, welches Attribut ein Key Attribut ist usw. und die Modelklasse direkt auf DynamoDB zu persistieren. F�r das vergleichsweise einfache Datenmodell dieses Projekts war dies nicht notwendig, aber f�r komplexere Projekte ist das ein gro�er Vorteil. \newline
Die Nutzung von DynamoDB war in diesem Projekt ebenfalls recht einfach. Es mussten zu jedem Temperaturwert ein Hashkey sowie ein Rangekey angegeben werden, die eigentlichen Temperaturwerte konnten dann direkt in diesem Datensatz gespeichert werden. Auch eine Map kann in DynamoDB so direkt und ohne Anpassung des Formats im Gegensatz zu relationalen Datenbanken in diesem Datensatz gespeichert werden . Zudem war DynamoDB in den Testphasen recht performant und konnte mehrere tausend Datens�tze pro Sekunde persistieren. Hier war es allerdings notwendig das Datenformat etwas anzupassen und pro Sensordurchlauf nur noch einen Datensatz zu erstellen und die einzelnen Temperaturwerte als Map in diesem Datensatz zu speichern und diesen Datensatz nur noch zu bearbeiten. Wurden alle Datens�tze einzeln persistiert, konnte DynamoDB nur noch einige hundert Temperaturen pro Sekunde maximal persistieren. Trotz gutem Datendurchsatz ist ein passendes Datenmodell also dennoch wichtig f�r eine gute Performance von DynamoDB. \newline
Ein negativer Punkt in Sachen Kinesis ist die permanente Speicherung der Daten innerhalb eines Kinesis Streams. Daten, die �ber einen Kinesis Stream gesendet werden, werden noch bis zu einer Woche gespeichert, damit sie weiterhin von einem Consumer gelesen werden k�nnen. Das ist zun�chst einmal ein positiver Punkt, allerdings ist die Speicherung vergleichsweise kostenintensiv. Wurde ein Stream nach Einsatz nicht gel�scht, wurden die �bertragenen Daten im Stream persistiert, was schon nach wenigen Tagen Kosten von mehreren Dollar zur Folge hatte. F�r dieses Projekt mit dieser Datenmenge kein Problem, da ja auch ein gewisses Budget gew�hrt worden war, bei gr��eren Projekten k�nnen dadurch aber unter Umst�nden gr��ere Kosten entstehen, die aber vollkommen unn�tig sind. Im Laufe dieses Projekts wurde keine M�glichkeit gefunden, die Persistierung der Daten in Streams schon im Vorhinein zu unterbinden, was das Monitoring der Amazon Web Services und besonders von Amazon Kinesis sehr wichtig machte, um keine unn�tigen Kosten zu verursachen. \newline
Positiv zu erw�hnen ist, dass die �bertragung der Daten �ber Kinesis in den Tests immer kostenlos war, da sich die Datenmenge noch im kostenlosen Bereich von Kinesis befand. Um innerhalb dieses Bereichs zu bleiben, wurden die Producer bewusst nur maximal einige Minuten betrieben. Bei dauerhaftem Einsatz w�rde die Datenmenge auch in den kostenpflichtigen Bereich steigen. Genauso waren auch die Kosten f�r die Nutzung von DynamoDB und Amazon EC2 im Cent- oder maximal im niedrigen Dollarbereich. \newline
Ein weiterer positiver Punkt ist die recht gute Dokumentation der verschiedenen Webservices von Amazon, die einen schnellen Einstieg erm�glicht. Die Grundlagen aller Services sind gut erkl�rt und auch die APIs und Client Libraries werden beschrieben. Besonders die Beispielprojekte einzelner Services sind ein guter Einstieg in AWS. Negativ ist, dass die Dokumentation �ber die Grundlagen meist nicht hinaus geht und im Bereich Amazon Kinesis die Kinesis Connectors \cite{kinesisconnector} nur sehr kurz beschrieben wurden, so dass diese in diesem Projekt nicht genutzt werden konnten, da es aufgrund der geringen Dokumentation zu viel Zeit gebraucht h�tte die Funktionsweise dieser Library zu verstehen.
\section{Evaluation AWS IoT}

\section{Evaluation anderer Implementierungsm�glichkeiten}

\section{Zusammenfassung}