\chapter{Fazit}

Das Projekt verlief insgesamt recht gut und es konnten alle entscheidenden Anforderungen an das Projekt umgesetzt werden. Die Evaluation der genutzten Amazon Web Services zeigte, dass sie f�r die Verarbeitung gr��erer Datenstr�me geeignet sind und auch bei gr��erem Datendurchsatz eine gute Performance bieten. Dies bezieht sich besonders auf AWS Kinesis, das die �bertragung der Temperaturdaten stark vereinfachte. Zudem zeigte sich auch eine gute Einsteigerfreundlichkeit bei der Nutzung der Client SDKs durch eine gute Dokumentation, die allerdings bei weiterf�hrenden Themen nicht immer so umfangreich war. Daher war es im Rahmen dieses Projektes leider nicht m�glich, einige Technologien zu nutzen, die weitere Erkenntnisse zu AWS h�tten bringen k�nnen. \newline
Positiv ist zudem die Projektplanung und insbesondere die Risikoplanung zu nennen, durch die einige Risiken abgeschw�cht werden konnten und das Projektziel somit nicht gef�hrdet wurde. Auch das Budget konnte dank der dauerhaften �berwachung der Kosten eingehalten werden, obwohl einige unerwartete Kosten anfielen. \newline
Die Integration des Projektes in ein Docker Image war dank der vorhandenen passenden Basisimages wie dem hier genutzten Maven Image ebenfalls sehr einfach und konnte gut umgesetzt werden. \newline
Zudem wurde in AWS IoT ein neuer Amazon Web Service evaluiert, der f�r die Nutzung von echten Sensoren konzipiert wurde und damit auch einen Blick in den potentiellen IoT-Markt der Zukunft erm�glicht. Hier kann ein generell positives Fazit gezogen werden, da AWS IoT einige Funktionen bietet, die die Kommunikation mit IoT-Ger�ten vereinfacht und die Integration von anderen Amazon Web Services zur Datenhaltung oder Verarbeitung der Daten erm�glicht. \newline
Zuk�nftig k�nnte dieses Projekt  entsprechend der bisher nicht umgesetzten Anforderungen erweitert werden oder die bisher umgesetzten Klassen k�nnten anders implementiert werden, um weitere Funktionen von Amazon Web Services zu testen.