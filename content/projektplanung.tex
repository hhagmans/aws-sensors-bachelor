Das Projekt wurde bereits in einer vorher bearbeiteten Projektarbeit \cite{projektarbeit} geplant. Dabei wurden mehrere Analysen durchgef�hrt, die in diesem Kapitel vorgestellt werden. Als Ergebnis der Anforderungsanalyse wurde folgende Liste von Anforderungen erstellt:

\begin{table}[h!]
	
	\begin{tabularx}{\textwidth}{|p{\textwidth/100*20}|p{\textwidth/100*50}|X|}
		
		\hline
		
		Thema & Beschreibung & Kano Bewertung \\
		
		\hline
		
		\hline
		
		Dauerhafter Datenstrom & Producer sollen dauerhaft Temperaturdaten liefern & Basismerkmal \\
		
		\hline
		
		Daten Persistenz & Die Daten sollen dauerhaft in einer Datenbank persistiert werden & Basismerkmal \\
		
		\hline
		
		Start und Stopp des Datenstroms & Der Datenstrom soll vom Nutzer gestartet und gestoppt werden k�nnen & Basismerkmal \\
		
		\hline
		
		Datenstrom Skalierung & Der Datenstrom soll skaliert werden k�nnen. Beispielsweise sollen die Daten eines Tages in einer Stunde ausgegeben werden k�nnen & Basismerkmal \\
		
		\hline
		
		Dashboard & Nutzer soll die aktuellen Daten in einem Dashboard einsehen k�nnen & Basismerkmal \\
		
		\hline
		
		Mehrere AWS Services & Die Applikation soll auf mehreren AWS Services deployed werden, um diese vergleichen zu k�nnen & Basismerkmal \\
		
		
		\hline
				Dashboard Darstellung & Die Daten werden in einem Dashboard dargestellt. Bspw. mittels Verlaufsdiagrammen und in tabellarischer Form. & Leistungsmerkmal \\
				
				\hline
		
	\end{tabularx}
	
	\caption{Anforderungen an das Projekt Teil 1}
	\label{anforderungen_1}
\end{table}

\begin{table}[h!]
	
	\begin{tabularx}{\textwidth}{|p{\textwidth/100*20}|p{\textwidth/100*50}|X|}
		
		\hline
		
		Thema & Beschreibung & Kano Bewertung \\
		
		\hline
		
		\hline
				
		Verf�gbarkeit & Die Services sollen hochverf�gbar sein & Leistungsmerkmal \\
		
		\hline
		
		Aktualit�t der Daten & Die Daten sollen im Dashboard aktuell gehalten werden, auch wenn die Seite nicht durch den Benutzer neu geladen wird & Begeisterungs-merkmal \\
		
		\hline
		
		Temperaturen beeinflussen sich gegenseitig & Die Temperaturen sollen sich gegenseitig beeinflussen, z.B. bedingt eine h�here Au�entemperatur eine h�here Innentemperatur & Leistungsmerkmal \\
		
		\hline
		
	\end{tabularx}
	
	\caption{Anforderungen an das Projekt Teil 2}
	\label{anforderungen_2}
\end{table}

Die Anforderungsliste ist nach der Priorisierung sortiert, die im Rahmen einer Kano Bewertung der einzelnen Punkte erstellt wurde. Wie in den Tabellen \ref{anforderungen_1} und \ref{anforderungen_2} zu sehen, wurden die Basismerkmale am h�chsten priorisiert, da diese den Erfolg des Projektes ausmachen. Auf der Basis dieser Anforderungsanalyse wurde zudem folgendes Pflichtenheft erstellt:

\begin{table}[h!]

\begin{tabularx}{\textwidth}{|p{\textwidth/100*30}|p{\textwidth/100*50}|X|}

\hline

Thema & Beschreibung & Aufwand \\

\hline

\hline

AWS kennenlernen & Kennenlernen der AWS Services und erste Deployments von Containern & 10 \\

\hline

Docker Image & Docker Image mit allen ben�tigten Ressourcen erstellen & 3 \\

\hline

Docker Container & Docker Container aus dem Image mit der fertigen Applikation erzeugen & 3 \\

\hline

Producer & Es m�ssen mehrere Producer geschrieben werden, die konstant Temperaturdaten liefern. Die Producer werden in Java geschrieben & 20 \\

\hline

Consumer & Es muss mindestens ein Consumer geschrieben werden, der die Daten der Producer verarbeitet. Der Consumer wird in Java implementiert & 20 \\

\hline

\end{tabularx}

\caption{Pflichtenheft Teil 1}

\end{table}

\begin{table}[h!]

\begin{tabularx}{\textwidth}{|p{\textwidth/100*30}|p{\textwidth/100*50}|X|}
	
	\hline
	
	Thema & Beschreibung & Aufwand \\
	
	\hline
	
	\hline
	
	Consumer DB & Es muss eine Datenbank entweder auf Amazon RDS oder Amazon DynamoDB (oder andere L�sungen bspw. mit Cassandra) eingerichtet werden & 10 \\
	
	\hline
	
	Consumer DB Zugriff & Der Consumer muss die Temperaturdaten in die DB schreiben k�nnen & 10 \\
	
	\hline
	
	Mehrere AWS Services & Die Applikation f�r mehrere AWS Services kompatibel machen und auf mehreren Services deployen & 20 \\
	
	\hline
	
	Dashboard & Es muss ein Dashboard geschrieben werden, das die Temperaturdaten anzeigen kann & 15 \\
	
	\hline
	
	Dashboard Start Stopp & Es muss im Dashboard die Funktion geben den Datenstrom anhalten oder wieder starten zu k�nnen & 3 \\
	
	\hline
		
	Dashboard Zeitskalierung & Es muss im Dashboard die Funktion geben den Datenstrom verschnellern oder verlangsamen zu k�nnen & 7 \\
	
	\hline
	
	Dashboard Diagramme & Die Daten sollten im Dashboard in Form von Diagrammen dargestellt werden & 10 \\
	
	\hline
	
	Dashboard Diagramme Aktualit�t & Die Daten sollten im Dashboard immer aktuell gehalten werden, auch wenn der Nutzer die Seite nicht aktualisiert & 5 \\
	
	\hline
	
	Producer Temperatur Beeinflussung & Die Temperaturwerte der Producer m�ssen sich gegenseitig beeinflussen. Die Producer m�ssen also untereinander kommunizieren und zumindest Wechsel in der Temperaturtendenz interessierten anderen Producern mitteilen. & 10 \\
	
	\hline
	
\end{tabularx}

\caption{Pflichtenheft Teil 2}

\end{table}

Das Pflichtenheft ist genau wie die Anforderungsliste nach der Priorisierung sortiert, die im Rahmen einer Kano Bewertung der einzelnen Punkte erstellt wurde. Genauere Erl�uterungen der einzelnen Anforderungen und Punkte des Pflichtenhefts finden sich in der Projektarbeit. \newline
Des Weiteren wurde im Rahmen der Projektplanung eine Risikoanalyse durchgef�hrt, um die gr��ten Risiken des Projekts zu erkennen und dementsprechende Gegenma�nahmen anwenden zu k�nnen. Hierbei haben sich folgende 4 Risiken herausgestellt:

\begin{table}[h!]
	
	\begin{tabularx}{\textwidth}{|p{\textwidth/100*10}|p{\textwidth/100*40}|p{\textwidth/100*15}|p{\textwidth/100*12}|X|}
		
		\hline
		
		Nummer & Risiko & Eintrittswahr-scheinlichkeit in \% & gesch�tzter Schaden in \texteuro & Risiko-faktor  \\
		
		\hline
		
		\hline
		
		1 & AWS Zugriff zu sp�t bekommen & 30 & 300 & 90 \\
		
		\hline
		
		2 & Producer erzeugt zu viele Daten und damit zu viele Kosten bei AWS & 50 & 150 & 75 \\
		
		\hline
		
		3 & Teile des Projekts werden nicht rechtzeitig vor Abgabe erstellt & 10 & 1000 & 100 \\
		
		\hline
		
		4 & Untersch�tzen des Umfangs oder der Schwierigkeit des Projektes & 10 & 1000 & 100 \\
		
		\hline
		
	\end{tabularx}
	
	\caption{Risikoliste}
	
\end{table}

Der Risikofaktor entspricht der Formel \(\frac{Eintrittswahrscheinlichkeit \cdot Schaden}{100}\).Dementsprechend befinden sich gerade die Risiken Nummer 3 und 4 durchaus in einem gef�hrlichen Bereich, der das Projekt gef�hrden k�nnte. In der Projektarbeit wurden allerdings Ma�nahmen zur Verhinderung des Eintretens der Risiken sowie der Verminderung der Kosten bei Eintreten der Risiken ermittelt, die in der Ausf�hrung des Projektes auch umgesetzt wurden (s. Kapitel \ref{abweichungen}).  \newline
Abschlie�end wurde eine Kostenplanung erstellt, weil die Nutzung von Amazon Web Services Kosten verursachen kann und diese nicht zu hoch ausfallen sollten. Es wurde ein Budget von bis zu 200  \texteuro{} gew�hrt, das nicht �berschritten werden sollte. Daher mussten die aktuellen Kosten st�ndig �berwacht werden. \newline
Im Rahmen der Projektarbeit wurden zudem die Themen und Komponenten, die Teil der Bachelorarbeit sein sollten, behandelt und beschrieben. Darunter z�hlten die Themen Cloud Computing, Amazon Web Services und deren einzelne Komponenten sowie Docker. Daher m�ssen diese Themen in dieser Bachelorarbeit nicht mehr ausf�hrlich behandelt werden. \newline
Eine erste Evaluation der in Frage kommenden Amazon Web Services wurde ebenfalls durchgef�hrt und dabei eine Kombination aus Amazon EC2 als Infrastruktur f�r die Producer und Consumer, Amazon Kinesis als �bertragungskanal f�r die Temperaturdaten sowie Amazon RDS oder Amazon DynamoDB f�r die Persistierung der Daten als passende Services ausgemacht.
