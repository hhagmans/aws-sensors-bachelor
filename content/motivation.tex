\section{Motivation}

Cloud Computing ist ein immer gr��er werdendes Thema in der IT. Immer mehr Daten werden nicht mehr lokal, sondern in der Cloud gespeichert und werden somit f�r jeden �berall verf�gbar. Auch Anwendungen werden immer h�ufiger nicht mehr lokal betrieben, sondern nutzen die gro�en Rechnerleistungen von Cloud L�sungen, um m�glichst skalierbar zu sein und mittels Redundanz h�here Verf�gbarkeiten und bessere Latenzzeiten zu erreichen. Zudem bieten Cloud L�sungen die M�glichkeit, Investitionskosten in Betriebskosten umzuwandeln, da keine neuen Server und andere Ger�te gekauft werden m�ssen, sondern f�r die Nutzung bezahlt wird. Webapplikationen bilden hier keine Ausnahme. Anstatt eigene Rechenzentren aufbauen zu m�ssen, werden Webapplikationen immer h�ufiger bei externen Cloudanbietern betrieben um Kosten und den Administrationsaufwand solcher Systeme zu reduzieren. Ein gro�er Anbieter solcher Plattformen ist Amazon, die mit ihren Amazon Web Services \cite{aws} eine Reihe von Cloud Services bieten, die von vielen weltweit operierenden Unternehmen genutzt wird. Doch wie kann man mittels Amazon Web Services gro�e Datenfl�sse wie beispielsweise die Aufzeichnung von Sensordaten in einem automatisierten Heimsystem am besten verwalten?

\section{Zielsetzung}
Es soll eine Wettersimulation erstellt werden, die aus mehreren Services besteht, die dauerhaft Daten liefern. Beispielsweise mehrere virtuelle Temperatursensoren, die Zufallszahlen innerhalb eines bestimmten Werteintervalls liefern. Die Temperaturwerte k�nnen sich auch gegenseitig beeinflussen. So beeinflusst eine h�here Au�entemperatur auch die Innentemperatur. Diese Komponenten werden jeweils in einem Docker Container \cite{docker} auf Amazon ECS \cite{ecs} deployed und liefern einen konstanten Datenstrom von Temperaturdaten. Der Datenstrom soll skaliert werden k�nnen und bspw. die Daten eines ganzen Tages in einer Stunde erzeugen. Die Daten werden in einer SQL Datenbank auf Amazon RDS \cite{rds} oder Amazon DynamoDB \cite{dynamodb} gespeichert. Eine eigene L�sung mit Apache Cassandra \cite{cassandra} w�re auch m�glich. Im Dashboard soll eine �bersicht der Daten in Verlaufsdiagrammen verf�gbar sein. Hier ist auch eine Anpassung der Zeitskalierung m�glich.
Der Fokus liegt hierbei nicht auf einer akkuraten Wettersimulation, sondern auf der Evaluation der AWS Services f�r gro�e Datenstr�me und der Vergleich zwischen den einzelnen AWS Services. Bspw. k�nnten Teile der Architektur auch mit Amazon Kinesis \cite{kinesis} umgesetzt werden, das speziell f�r gro�e Datenstr�me konzipiert wurde. Hier k�nnte man beide Architekturentw�rfe vergleichen.
Dieses Projekt wurde in einer Projektarbeit bereits geplant und analysiert und soll in dieser Bachelorarbeit nun implementiert werden. Daraufhin soll eine Evaluation der verwendeten AWS Komponenten durchgef�hrt werden.

\section{Vorgehensweise}
\begin{itemize}
	\item Wie wird vorgegangen, um das Ziel zu erreichen?
	\item Warum ist die Arbeit so gegliedert, wie sie gegliedert ist?
	\item Welche Aspekte werden nicht behandelt \textbf{und} warum?
\end{itemize}